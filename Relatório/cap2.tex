\chapter{Planejamento e implementação}\label{cap-planejamento-implementacao}

\section{Problema A}\label{problemaA}
\begin{enumerate}[label=\textbf{\alph*)},series=problemas]
\item Calcular a distância percorrida por um determinado robô ao longo do processo de resgate das vítimas. Considere que a distância total percorrida deve ser calculada como a soma de todas as distâncias entre os pontos de passagem do robô.
\end{enumerate}

\noindent \textbf{COMPREENSÃO DO PROBLEMA E PLANEJAMENTO}: deve-se extrair a tupla de um robô da lista de entrada, de acordo com o identificador dado. Em seguida, calcular a distância total que o robô percorreu e retornar esse valor.

Sendo assim, podemos criar as seguintes funções para solucionar o problema:
\begin{enumerate}
    \item Função para somar todos os elementos de uma lista
    \item Função para cálculo de distância no plano cartesiano
    \item Função para extrair uma tupla de uma lista dado um elemento identificador da tupla
    \item Função para calcular a distância entre os pontos percorridos pelo robô, por ordem de instante\label{itemDistInstante}
    \item Função para ordenar lista por instante (talvez não precise disso, precisamos ver o exemplo de entrada)
\end{enumerate}

As duas primeiras são triviais e foram implementadas diversas vezes durante o curso. Na terceira, utilizamos [...] para retornar a tupla que desejávamos. Para o item \ref{itemDistInstante}, [...].

\section{Problema B}\label{problemaB}
\begin{enumerate}[label=\textbf{\alph*)},resume*=problemas]
\item Determine qual dos robôs apresenta o seu último ponto de passagem no terreno de busca que possui a maior distância em relação à origem. Exiba o caminho percorrido pelo robô e o tempo total do percurso;
\end{enumerate}


% # - Extrair, da lista de entrada, o último ponto de passagem de todos os robôs
% # - Determinar qual robô possui ponto mais longe da origem
% # - Imprimir caminho percorrido pelo robô, determinar tempo do percurso

\section{}