\chapter*{Introdução}\label{cap-introducao} % Basicamente finalizado

Este trabalho tem por objetivo documentar a estruturação de um projeto de auxílio à tomada de decisões dado um cenário hipotético de um ambiente acidentado. Neste, dados serão extraídos por robôs-célula que investigam o local, que serão processados de forma a aprimorar a atividade de resgate das vítimas que se encontram no ambiente. O plano de testes, presentes no Capítulo~\ref{cap-planejamento-implementacao}, relata a confiabilidade das funções desenvolvidas.

As referências para os métodos de testes, bem como paradigmas aplicativo e recursivo, que serão implementados podem ser encontrados nas notas de aula utilizadas pela disciplina encontradas na página desta em maio de 2019.

Além do plano de testes, no Capítulo~\ref{cap-planejamento-implementacao} também abordamos a forma como planejamos a resolução de cada problema apresentado pelo trabalho, assim como a implementação, de forma geral, das funções mais complexas. Em seguida, o Capítulo~\ref{cap-analise-resultados} realiza uma avaliação da solução final e discorre acerca dos resultados obtidos, também pontuando comentários adicionais.