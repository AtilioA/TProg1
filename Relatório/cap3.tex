\chapter*{CONCLUSÃO}\label{cap-conclusao}
\addcontentsline{toc}{chapter}{CONCLUSÃO}

% Discutir as conclusões e limitações do seu trabalho.

Pelo estudo realizado neste trabalho, fica evidente como podemos construir diversas soluções tendo como ponto de partida o paradigma funcional. Utilizando a modularização e o reuso de funções através de importações, foi possível conceber uma central de processamento fictícia adaptada para o contexto proposto ao projeto. Ademais, a compreensão de problemas e planejamento de soluções permitiram que o desenvolvimento das resoluções fosse muito mais focado e centrado. 

Os conceitos de paradigma recursivo e aplicativo, além da compreensão de lista foram fundamentais para a execução das soluções idealizadas. A utilização dessas abstrações permitiu uma solução mais limpa e sucinta do que se fosse empregado o paradigma procedural, por exemplo. O uso de recursão, no entanto, pode causar \textit{stack overflow} e, embora a implementação do \textit{merge sort} seja significativamente mais simples neste paradigma, um grande volume de dados poderia causar a suspensão da execução do programa; o próprio Python limita o nível de recursões a 1000 chamadas empilhadas por padrão, embora seja possível aumentar esse número.