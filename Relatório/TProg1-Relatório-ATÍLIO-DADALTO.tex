\documentclass[
	% --- opções da classe memoir ---
	12pt,				% Tamanho da fonte
	openright,			% Capítulos começam em pág ímpar (insere página vazia caso preciso)
	twoside,			% Para impressão em recto e verso. Oposto a oneside
	a4paper,			% Tamanho do papel.
	% -- opções do pacote babel --
	english,			% Idioma adicional para hifenização
	french,				% Idioma adicional para hifenização
	spanish,			% Idioma adicional para hifenização
	brazil,				% O último idioma é o principal do documento
	]{abntex2}


% ---
% Pacotes
% ---
\usepackage{lmodern}			% Usa a fonte Latin Modern
\usepackage[T1]{fontenc}		% Seleção de códigos de fonte.
\usepackage[utf8]{inputenc}		% Codificação do documento (conversão automática dos acentos)
\usepackage{indentfirst}		% Indenta o primeiro parágrafo de cada seção.
\usepackage{color}				% Controle das cores
\usepackage{graphicx}			% Inclusão de gráficos
\usepackage{microtype} 			% Para melhorias de justificação
\usepackage{float}
\usepackage{mathabx}
\usepackage[ampersand]{easylist}

\usepackage{multicol}
\usepackage{multirow}

\usepackage{lipsum}

\usepackage[brazilian,hyperpageref]{backref}
\usepackage[alf]{abntex2cite}


% ---
% CONFIGURAÇÕES DE PACOTES
% ---

% ---
% Configurações do pacote backref
% Usado sem a opção hyperpageref de backref
\renewcommand{\backrefpagesname}{Citado na(s) página(s):~}
% Texto padrão antes do número das páginas
\renewcommand{\backref}{}

% ---
% Informações de dados para CAPA e FOLHA DE ROSTO
% ---
\titulo{Implementação de uma Unidade Lógica Aritmética com Portas Lógicas Básicas}
\autor{Atílio Antônio Dadalto \\ Vitor Ferraz Matos Brunoro}
\local{Vitória}
\data{2018}
\instituicao{%
  Universidade Federal do Espírito Santo
  \par Departamento de Informática}
\tipotrabalho{Relatório}
\preambulo{Relatório apresentado como requisito parcial para aprovação na disciplina de Elementos de Lógica Digital, pela Universidade Federal do Espírito Santo.muito bom}

\usepackage[LGRgreek]{mathastext}

% Alterando o aspecto da cor azul
\definecolor{blue}{RGB}{41,5,195}

% Informações do PDF
\makeatletter
\hypersetup{
     	%pagebackref=true,
		pdftitle={\@title},
		pdfauthor={\@author},
    	pdfsubject={\imprimirpreambulo},
	    pdfcreator={LaTeX with abnTeX2},
		pdfkeywords={abnt}{latex}{abntex}{abntex2}{relatório técnico},
		colorlinks=true,       		% False: boxed links; true: colored links
    	linkcolor=blue,          	% Color of internal links
    	citecolor=blue,        		% Color of links to bibliography
    	filecolor=magenta,      		% Color of file links
		urlcolor=blue,
		bookmarksdepth=4
}
\makeatother

% O tamanho do parágrafo é dado por:
\setlength{\parindent}{1.3cm}

% Controle do espaçamento entre um parágrafo e outro:
\setlength{\parskip}{0.2cm}  % Tente também \onelineskip

% ---
% Início do documento
% ---
\begin{document}

% Seleciona o idioma do documento (conforme pacotes do babel)
\selectlanguage{brazil}

% Retira espaço extra obsoleto entre as frases.
\frenchspacing

% ---
% Capa
% ---
\imprimircapa

\imprimirfolhaderosto

\tableofcontents* % Não exibe o sumário no sumário

% ----------------------------------------------------------
% ELEMENTOS TEXTUAIS
% ----------------------------------------------------------
\textual

% ----------------------------------------------------------
% COMEÇO DO TEXTO
% ----------------------------------------------------------
\chapter*[Introdução]{Introdução}
\addcontentsline{toc}{chapter}{Introdução}

\lipsum[10]

Neste projeto, buscamos implementar todas as funções necessárias para a composição de uma Unidade Lógica e Aritmética que opere em 16 bits, além de uma calculadora com seu próprio display hexadecimal de saída, tomando como ferramenta o software utilizado durante o curso, \textit{Logisim}.

Através da modularidade, foi possível utilizar a abordagem de dividir para conquistar, tornando o projeto como um todo mais organizado e manutenível. Isso provou-se notadamente útil na construção do multiplexador 8:1 com entrada de 8 bits (\autoref{mux818}), por exemplo.

Este relatório documenta a trajetória da construção dessa Unidade Lógica e Aritmética através de portas lógicas básicas, de duas entradas, pontuando as sub-funções elaboradas e como foram integradas, bem como os testes efetuados, estes no \autoref{apendiceA}.

\chapter{A Unidade Lógica e Aritmética}

\lipsum[11]

\chapter{Calculadora}

\lipsum[12]

\chapter{A ULA (16 bits)}

\lipsum[13]

\chapter*[Conclusão]{Conclusão}
\addcontentsline{toc}{chapter}{Conclusão}

Pelo estudo realizado neste trabalho, fica evidente como podemos chegar a sistemas progressivamente mais complexos, como uma Unidade Lógica e Aritmética, tendo como ponto de partida portas lógicas básicas. Iniciamos o projeto com portas lógicas AND, OR e NOT, de duas entradas, para criar os circuitos aritméticos como o somador completo e o subtrator completo, e, a partir dessas estruturas, utilizamos a modularização e o reuso desses circuitos como caixas pretas para conseguir operar números binários de mais algarismos. Em seguida, também com as portas lógicas básicas, podemos criar as operações lógicas AND, OR e XOR bit a bit, além das instruções de SHIFT LEFT e SHIFT RIGHT.

Posteriormente, apenas com o uso de multiplexadores igualmente construídos com portas lógicas básicas, foi possível integrar todas as operações supracitadas, estruturando, portanto, uma Unidade Lógica e Aritmética de 8 bits. Com esta e o uso de decodificadores, foi possível a implementação de uma calculadora com saída que representa dois dígitos hexadecimais em displays de 7 segmentos. Por outro enfoque, mas lançando mão dos mesmos conceitos, foi possível utilizar a ULA de 8 bits para implementar uma ULA de 16 bits.

\begin{apendicesenv}

\chapter{Testes dos circuitos}\label{apendiceA}

Este apêndice serve como repositório para os testes dos circuitos principais utilizados no projeto.

\end{apendicesenv}


\end{document}
